\section{ЗАДАЧА 1}

Вероятность некоторого случайного события равна 0.9. Какова вероятность того, что после 64 000 независимых испытаний наблюденная частота этого события лежит в интервале 0.9$\pm$0.01? Решить задачу, используя неравенство Чебышева и интегральную теорему Муавра-Лапласа

\subsection{Неравенство Чебышева}

Пусть

\begin{itemize}
    \item $X$ - случайная величина
    \item $\exists MX,\ \exists DX$
\end{itemize}

\begin{equation*}
    \forall \varepsilon > 0, P\{ |X - MX| \geq \varepsilon \} \le \frac{DX}{\varepsilon^2}
\end{equation*}

\textbf{Решение}

\begin{itemize}
    \item Пусть $k_n$ -- число успехов в серии испытаний по схеме Бернулли ($p = 0.9$)
    \item $M[K_n] = np = 64000 \cdot 0.9 = 57600$
    \item $D[K_n] = npq = 64000 \cdot 0.9 \cdot 0.1 = 5760$
\end{itemize}

\begin{multline*}
    P \{ 0.89 \le r_\text{н} \le 0.91 \} = P \{ 0.89 \le \frac{k_n}{n} \le 0.91 \} = P \{ 56960 \le k_n \le 58240 \} = \\
    = P \{ -640 \le k_n - M[K_n] \le 640 \} = P\big\{ \big|k_n - M[K_n]\big| \le 640 \big\} \geq 1 - \frac{D[K_n]}{\varepsilon^2} = \\
    = 1 - \frac{5760}{640^2} = 1 - 0.014 = 0.986
\end{multline*}

\textbf{Ответ:} $P \{ 0.89 \le r_\text{н} \le 0.91 \} \geq 0.986$

\subsection{Теорема Муавра-Лапласа}

\textbf{Решение}

Имеем схему испытаний Бернулли

\begin{itemize}
    \item $n = 64000 \gg 1$
    \item $p = 0.9$
    \item $q = 1 - p = 0.1$
\end{itemize}

\begin{multline*}
    P \{ 0.89 \le r_\text{н} \le 0.91 \} = P \{ 0.89 \le \frac{k_n}{n} \le 0.91 \} = P \{ 56960 \le k_n \le 58240 \} \approx \\
    \approx \Phi_0(x_2) - \Phi_0(x_1) \approx 2 \Phi_0 (8.4327) \approx 2 \cdot 0.49999 \approx 0.99998
\end{multline*}

\begin{equation*}
    x_1 = \frac{k_1 - np}{\sqrt{npq}} = \frac{56960 - 64000 \cdot 0.9}{\sqrt{64000 \cdot 0.9 \cdot 0.1}} \approx \frac{-640}{75.89} \approx -8.4327
\end{equation*}

\begin{equation*}
    x_2 = \frac{k_2 - np}{\sqrt{npq}} = \frac{58240 - 64000 \cdot 0.9}{\sqrt{64000 \cdot 0.8 \cdot 0.1}} \approx \frac{640}{75.89} \approx 8.4327
\end{equation*}

\textbf{Ответ:} $P \{ 0.89 \le r_\text{н} \le 0.91 \} \approx 0.99998.$

\section{ЗАДАЧА 2}

С использованием метода моментов для случайной выборки $\vec X = (X_1, ..., X_n)$ из генеральной совокупности $X$ найти точечные оценки указанных параметров заданного закона распределения.

\textbf{Закон распределения}

\begin{equation*}
    f_X(x) = \theta^2 xe^{-\theta x}, x > 0
\end{equation*}

\textbf{Решение}

\begin{enumerate}
    \item Неизвестный параметр $\theta$ $\Rightarrow$ $r = 1$ $\Rightarrow$ найдем моменты до первого порядка

        \begin{multline*}
            MX = \int_{-\infty}^{+\infty} x \cdot f_X(x) dx = \int_0^{+\infty} x \theta^2 xe^{-\theta x} dx = \theta^2 \int_{0}^{+\infty} x^2 e^{-\theta x} dx = \\
            = \theta^2 \cdot \bigg( - \frac{1}{\theta} \bigg) \int_0^{+\infty} x^2 de^{-\theta x} = \underbrace{-\theta x^2 e^{-\theta x} \bigg|_0^{+\infty}}_{0} + 2\theta \int_0^{+\infty} xe^{-\theta x} dx = \\
            = -2 \int_0^{+\infty} xde^{-\theta x} = \underbrace{-2 xe^{-\theta x} \bigg|_0^{+\infty}}_{0} + 2 \int_0^{+\infty} e^{-\theta x} dx = -\frac{2}{\theta} e^{-\theta x} \bigg|_0^{+\infty} = \frac{2}{\theta}
        \end{multline*}

    \item Приравняем теоретические моменты к их выборочным аналогам

        \begin{equation*}
            \frac{2}{\theta} = \overline X \Rightarrow \theta = \frac{2}{\overline X}
        \end{equation*}

        \begin{equation*}
            \overline X = \frac{1}{n} \sum_{i=1}^n X_i
        \end{equation*}
\end{enumerate}

\textbf{Ответ:} $\theta = \frac{2}{\overline X}$

\section{ЗАДАЧА 3}

С использованием метода максимального правдоподобия для случайной выборки $\vec X = (X_1, ..., X_n)$ из генеральной совокупности $X$ найти точечные оценки параметров заданного закона распределения. Вычислить выборочные значения найденных оценок для выборки $\vec x_5 = (x_1, ..., x5)$.

\textbf{Закон распределения}

\begin{equation*}
    f_X(x) = \frac{1}{\theta^2} xe^{-x / \theta}, x > 0
\end{equation*}

\textbf{Выборка $\vec x_5$}

\begin{equation*}
    (4, 12, 6, 7, 9)
\end{equation*}

\textbf{Решение}

\begin{equation*}
    L(\vec X, \theta) = f(X_1, \theta) \cdot ... \cdot f(X_n, \theta) = \frac{1}{\theta^{2n}} \cdot \big( X_1 \cdot ... \cdot X_n \big) \cdot \big( e^{-X_1/\theta} \cdot ... \cdot e^{-X_n/\theta} \big)
\end{equation*}

\begin{equation*}
    \ln L = -2n \ln \theta + \ln \big(X_1 \cdot ... \cdot X_n \big) - \frac{X_1 + ... + X_n}{\theta}
\end{equation*}

Необходимое условие экстремума

\begin{equation*}
    \frac{\partial \ln L}{\partial \theta} = \frac{-2n}{\theta} + \frac{X_1 + ... + X_n}{\theta^2} = 0
\end{equation*}

\begin{equation*}
    \frac{X_1 + ... + X_n}{\theta} = 2n \Rightarrow \theta = \frac{X_1 + .. + X_n}{2n} = \frac{\overline X}{2}
\end{equation*}

Достаточное условие экстремума

\begin{equation*}
    \frac{\partial^2 \ln L}{\partial^2 \theta} = \frac{2n}{\theta^2} - 2 \frac{X_1 + ... + X_n}{\theta^3} = \frac{2n}{\theta^2} - \frac{2n \overline X}{\theta^3} = \frac{8n}{\overline X^2} - \frac{16n}{\overline X^2} < 0
\end{equation*}

\begin{equation*}
    \theta = \frac{\overline X}{2} \text{ -- точка локального максимума}
\end{equation*}

Подставим выборку $\vec x_5$

\begin{equation*}
    \theta = \frac{4 + 12 + 6 + 7 + 9}{2 \cdot 5} = \frac{38}{10} = 3.8
\end{equation*}

\section{ЗАДАЧА 4}

При помощи вольтметра, точность которого характеризуется средним квадратичным отклонением $\sigma = 0.2$ В, произведено $n = 10$ измерений напряжения бортовой батареи, в результате которых получено $X_n = 50.2$ В. Считая распределение контролируемого признака нормальным, построить для него доверительный интервал уровня $\gamma = 0.95$.
