\section{ЗАДАЧА 1}

\subsection{Проверка параметрических гипотез}

Известно, что точность манометра характеризуется средним квадратичным отклонением $\sigma = 1$ Па. В результате $n_1 = 5$ измерений давления в пневмосистеме ракеты было получено среднее значение $\overline{x}_{n_1} = 150$ Па. После шестимесячного хранения ракеты давление в пневмосистеме было измерено $n_2 = 3$ раза, в результате чего было получено значение $\overline{y}_{n_2} = 148$ Па. Считая, что случайные погрешности измерений подчинены нормальному закону, при уровне значимости $\alpha = 0.05$ проверить гипотезу о том, что за время хранения давление в пневмосистеме ракеты не изменилось.

\textbf{Решение}

Пусть слачайная величина  $X$ -- результат первых измерений. $Y$ -- результат измерений спустя пол года.

Предположим, что $X \sim N(m_1, \sigma^2)$ и $Y \sim N(m_2, \sigma^2)$, причем $\sigma = 1$, $m_1 = MX$, $m2 = MY$.

Введем основную гипотезу

\begin{equation*}
    H_0 = \{ \text{давление не изменилось} \} = \{m_1 = m_2\}
\end{equation*}

С учетом средних значений экспериментов введем конкурирующую гипотезу:

\begin{equation*}
    H_1 = \{ \text{давление уменьшилось} \} = \{ m_1 > m_2 \}
\end{equation*}

Используем статистику

\begin{equation*}
    T \big( \vec X_{n_1}, \vec Y_{n_2} \big) = \frac{\overline X - \overline Y}{\sqrt{\frac{\sigma_1^2}{n_1}+\frac{\sigma_2^2}{n_2}}} \sim N(0;1)
\end{equation*}

При истинности гипотезы $H_0$ построим критическое множество:

\begin{equation*}
    W = \big\{ \big( \vec x, \vec y \big) : T \big(\vec x, \vec y \big) \geq u_{1 - \alpha}\big\}
\end{equation*}

\begin{equation*}
    T\big( \vec x, \vec y \big) = \frac{150 - 148}{\sqrt{\frac{1}{5} + \frac{1}{3}}} = \frac{2 \cdot \sqrt{15}}{\sqrt{8}} \approx 2.739
\end{equation*}

\begin{equation*}
    u_{1- \alpha} = u_{0.95} = 1.645
\end{equation*}

\begin{equation*}
    2.739 \geq 1.645 \Rightarrow \big( \vec x, \vec y \big) \in W \Rightarrow \text{принимается гипотеза } H_1, \text{отклоняется } H_0
\end{equation*}

\textbf{Ответ}: давление изменилось.
